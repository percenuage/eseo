%%%%%%%%%%%%%%%%%%%%%%%%%%%%%%%%%%%%%%%%%%%
%	 Index, Glossaires et Nomenclature	  %
%%%%%%%%%%%%%%%%%%%%%%%%%%%%%%%%%%%%%%%%%%%
%\usepackage{makeidx}
	%\makeindex
%\usepackage[toc]{glossaries} %[nonumberlist,xindy]
	%\makeglossaries
%\usepackage[french,intoc]{nomencl}
	%\makenomenclature


%%%%%%%%%%%%%%%%%%%%%%%%%%%%%%%%%%%%%%%%%%%
%		  		  Langue     			  %
%%%%%%%%%%%%%%%%%%%%%%%%%%%%%%%%%%%%%%%%%%%
\usepackage[utf8]{inputenc} 
\usepackage[T1]{fontenc}
\usepackage[french]{babel} % Langue française
\usepackage{lmodern} %[bookman, charter, newcent, mathpazo, mathptmx, lmodern]


%%%%%%%%%%%%%%%%%%%%%%%%%%%%%%%%%%%%%%%%%%%
%	     Gestion couleurs LaTeX  	 	  %
%%%%%%%%%%%%%%%%%%%%%%%%%%%%%%%%%%%%%%%%%%%
\usepackage{color} %Couleur dans latex
\usepackage[svgnames]{xcolor} %Plus de couleurs
	%\definecolor{nomCouleur}{rgb}{0,0,0}
%\usepackage{colorbl} %Colorer du texte dans un tableau

%%%%%%%%%%%%%%%%%%%%%%%%%%%%%%%%%%%%%%%%%%%
%	  		Gestion des liens 	 		  %
%%%%%%%%%%%%%%%%%%%%%%%%%%%%%%%%%%%%%%%%%%%
\usepackage{url} %affichage d'url
\usepackage{hyperref} % hyperlien + toc
	\hypersetup{pdfborder={0 0 0}} %Supprime le cadre
	%%\href{url}{textLien} %no style url
	%%\nolinkurl{url} %style url
	%%\href{mailto:truc@bidon.org}{truc@bidon.org}
	%%\href{run:/cheminvers/fichier.pdf}{voir ce pdf}


%%%%%%%%%%%%%%%%%%%%%%%%%%%%%%%%%%%%%%%%%%%
%	  	 Gestion des filigranes 	 	  %
%%%%%%%%%%%%%%%%%%%%%%%%%%%%%%%%%%%%%%%%%%%
%\usepackage{draftwatermark} % [firstpage]
	%\SetWatermarkLightness{0.7}
	%\SetWatermarkColor{red} 
	%\SetWatermarkColor[rgb]{0,0,1}
	%\SetWatermarkAngle{25}
	%\SetWatermarkScale{2}
	%\SetWatermarkFontSize{2cm}
	%\SetWatermarkText{Texte en filigrane}


%%%%%%%%%%%%%%%%%%%%%%%%%%%%%%%%%%%%%%%%%%%
%	     Gestion des marges LaTeX	 	  %
%%%%%%%%%%%%%%%%%%%%%%%%%%%%%%%%%%%%%%%%%%%
\usepackage{geometry} %Modification grossière des marges
%\usepackage{fullpage}
\usepackage{layout} %Permet d'afficher toute les marges "\layout"
%  1 inch= 72pt = 2.54cm

%\oddsidemargin = 27pt %pair
%\evensidemargin = 8pt %impair
%\textwidth = 418pt
%\marginparwidth = 0pt
%\topmargin = 50pt
\headheight = 15pt
%\textheight = 640pt
%\headsep = 10pt

%Hauteur disponible : 21,8 cm
%Largeur disponible : 16,2 cm



%%%%%%%%%%%%%%%%%%%%%%%%%%%%%%%%%%%%%%%%%%
%	   Gestion des figures 				 %
%%%%%%%%%%%%%%%%%%%%%%%%%%%%%%%%%%%%%%%%%%
%\usepackage{wrapfig} %Figure incluse dans un texte
%\usepackage{subcaption}  %Sous-figure (subfig et subfigure déprécié)
%\usepackage{caption}  %Requis par subcaption
\usepackage{placeins} %Enpêche les figures de dépasser la barrier donnée
\usepackage{graphicx} %Introduction de document exterieur (png, jpg)
\usepackage{epstopdf} 


%%%%%%%%%%%%%%%%%%%%%%%%%%%%%%%%%%%%%%%%%%%
%  		 Mathématiques sous LaTeX		  %
%%%%%%%%%%%%%%%%%%%%%%%%%%%%%%%%%%%%%%%%%%%
%\usepackage{amsmath} %Package formule mathématiques
%\usepackage{amsfonts}
%\usepackage{amssymb}
%\usepackage{amsthm} %théorème


%%%%%%%%%%%%%%%%%%%%%%%%%%%%%%%%%%%%%%%%%%%
%  		 Physiques sous LaTeX		      %
%%%%%%%%%%%%%%%%%%%%%%%%%%%%%%%%%%%%%%%%%%%
%\usepackage{amsmath} %Package formule mathématiques
%\usepackage{amsfonts}
%\usepackage{amssymb}
%\usepackage{schemabloc}

%http://fr.wikibooks.org/wiki/LaTeX/%C3%89crire_de_la_physique
%http://fr.wikibooks.org/wiki/LaTeX/%C3%89crire_des_formules_chimiques
%http://sciences-indus-cpge.papanicola.info/Schema-blocs-avec-PGF-TIKZ-sous 
	%PGF/TIKZ : schéma-bloc

%%%%%%%%%%%%%%%%%%%%%%%%%%%%%%%%%%%%%%%%%%%
%	    Informatiques sous LaTeX  	      %
%%%%%%%%%%%%%%%%%%%%%%%%%%%%%%%%%%%%%%%%%%%	

%\usepackage{texments} %couleur du code ?
%\usepackage{moreverb} %numero de ligne ?
%\usepackage{verbatim}
\usepackage{listings} %no with utf8 %%\lstset
%\usepackage{minted}


%%%%%%%%%%%%%%%%%%%%%%%%%%%%%%%%%%%%%%%%%%%
%	  		Autres packages  	 		  %
%%%%%%%%%%%%%%%%%%%%%%%%%%%%%%%%%%%%%%%%%%%
\usepackage[final]{pdfpages} %Insère PDF 
	%%\includepdf[pages=1-10]{nom.pdf}
%\usepackage[toc,page]{appendix} %Annexe
	%%\appendix ou \begin{appendices}
%\usepackage{lettrine}	
	%%\lettrine{L}{suite}
%\usepackage{eurosym} %Symbole euro 
	%%\euro
%\usepackage{setspace} %Interligne
%\usepackage{soul} %Souligne/barre du texte
%\usepackage[tight]{shorttoc} % ou minitoc
	%%\shorttoc{titre}{niveau}
%\usepackage{multicol}
	%%\begin{multicols}{nb}
%\usepackage{minipage}
%\usepackage{enumitem}
	%%\begin{description}[font=\normalfont]
%\usepackage{lipsum}
	%%\lipsum[1-7] ou \lipsum ou \lipsum[5]
	%%\setlipsumdefault{n°1-n°2}
\usepackage{tikz-uml}


%%%%%%%%%%%%%%%%%%%%%%%%%%%%%%%%%%%%%%%%%%%
%   Gestion de la présentation LaTex	  %
%%%%%%%%%%%%%%%%%%%%%%%%%%%%%%%%%%%%%%%%%%%
%\usepackage[tikz]{bclogo} %Package de note dans les pages (avec logo)
%\usepackage{framed} %Package pour les notes classiques
\usepackage{sectsty} %Changement de l'affichage des sections latex
\usepackage{fancyhdr} %Entete et pied de page
\usepackage{lastpage} %Sort la dernière page du document "LastPage"


	\sectionfont{\color{RoyalBlue}{}} % Couleur des sections
	\subsectionfont{\color{DodgerBlue}{}} %Couleur des subsections

%Site pour d'autres couleur : http://en.wikibooks.org/wiki/LaTeX/Colors
